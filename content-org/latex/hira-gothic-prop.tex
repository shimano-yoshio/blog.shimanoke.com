% LuaLaTeX文書; UTF-8
\documentclass[14pt,a4paper]{ltjsarticle}

% 欧文フォントの設定-----
% https://ja.osdn.net/projects/luatex-ja/wiki/LuaTeX-ja%E3%81%AE%E4%BD%BF%E3%81%84%E6%96%B9
% http://myfuturesightforpast.blogspot.jp/2013/12/tex-gyre.html など
% 欧文
\usepackage[no-math]{fontspec}
% Avant Garde -> TeX Gyre Adventor
% Bookman Old Style -> TeX Gyre Bonum
% Zapf Chancery -> TeX Gyre Chorus
% Courier -> TeX Gyre Cursor
% Helvetica -> TeX Gyre Heros
% Helvetica Narrow -> TeX Gyre Heros Cn
% Palatino -> TeX Gyre Pagella
% New Century Schoolbook -> TeX Gyre Schola
% Times -> TeX Gyre Termes

\setmainfont[Ligatures=TeX]{TeXGyreTermes}
\setsansfont[Ligatures=TeX]{TeXGyreHeros}
\setmonofont[Scale=MatchLowercase]{TeXGyreCursor}
\renewcommand{\familydefault}{\sfdefault}

% 数式
\usepackage{amsmath}
\usepackage{bm}
% 数式用フォント
\usepackage{eulervm}

% 日本語フォント設定
\usepackage{luatexja-fontspec}
\setmainjfont[
  Script=Latin,
  YokoFeatures={JFM=myprop}, % カスタムのJFMを指定する
  CharacterWidth=Proportional,
  Kerning=On,
  BoldFont={HiraKakuProN-W6},
  BoldItalicFont={HiraKakuProN-W6},
  ItalicFont={HiraKakuProN-W6},
  BoldSlantedFont={HiraKakuProN-W6},
  BoldItalicFeatures={FakeSlant=0.2},
  BoldSlantedFeatures={FakeSlant=0.2},
]{HiraKakuProN-W3}

\setsansjfont[
  Script=Latin,
  YokoFeatures={JFM=myprop}, % カスタムのJFMを指定する
  CharacterWidth=Proportional,
  Kerning=On,
  BoldFont={HiraKakuProN-W6},
  BoldItalicFont={HiraKakuProN-W6},
  BoldSlantedFont={HiraKakuProN-W6},
  BoldItalicFeatures={FakeSlant=0.2},
  BoldSlantedFeatures={FakeSlant=0.2},
  ItalicFont={HiraKakuProN-W6},
]{HiraKakuProN-W3}

\usepackage[pdfencoding=auto]{hyperref}
\hypersetup{%
  bookmarksnumbered=true,%
  colorlinks=true,%
  pdftitle={タイトル},%
  pdfauthor={作成者},%
  pdfsubject={サブタイトル},%
  pdfkeywords={キーワード1\000\012キーワード2\000\015キーワード3\000\015\000\012キーワード4}}
% 
% 和欧文間空白を調整する
% \setxkanjiskip{0.1\zw minus 0.1\zw} % bxjclass
\ltjsetparameter{xkanjiskip=0.1\zw minus 0.1\zw}

\title{ヒラギノ角ゴシックによるプロポーショナル組版}
\author{島野善雄}

\begin{document}
\maketitle
\tableofcontents

\begin{abstract}
  ヒラギノ角ゴシックでプロポーショナルで組版する試みです。
\end{abstract}

\section{UTF-8 文字}
ゆきだるま☃は素敵・☃・①・鷗・を出したい。

\begin{enumerate}
\item 白鴎と白鷗
\item「吉野家」と「𠮷野家」
\item 森鷗外と内田百閒とが髙島屋に行くところを想像した。
\item \textbf{葛飾区の𠮷野家}
\item Macintosh用キーボードの⌘(Command key)を押す。
\item ♲ を心がけよう。
\end{enumerate}

\section{数式}

\begin{verbatim}
\begin{equation} % テスト
  \label{eq:1}
  y = a x^2 + b
\end{equation}
\end{verbatim}

\begin{equation}
  \label{eq:1}
  y = a x^2 + b
\end{equation}

\begin{equation}
    y = \sin x  \label{eq:1}
\end{equation}

\begin{equation}
    y = \cos x  \label{eq:2}
\end{equation}

後半の数式~\eqref{eq:2} だけに数式番号が付く.

\begin{equation}
    \left.
    \begin{aligned}
        u_t + (u \cdot \nabla)u + \nabla p &= f \\
        \operatorname{div} u               &= 0
    \end{aligned}
    \right\} 
    \quad \text{Euler equation}
\end{equation}

\section{吉野家コピペ・日本語}
 
昨日、近所の吉野家行ったんです。\emph{吉野家}。 
そしたらなんか人がめちゃくちゃいっぱいで座れないんです。 
で、よく見たらなんか垂れ幕下がってて、「150円引き」とか書いてあるんです。 
もうね、アホかと。馬鹿かと。 
お前らな、150円引き如きで普段来てない吉野家に来てんじゃねーよ、ボケが。 
150円だよ、150円。 
なんか親子連れとかもいるし。一家4人で吉野家か。おめでてーな。 
\emph{「よーしパパ特盛頼んじゃうぞー」}とか言ってるの。もう見てらんない。 
\textbf{お前らな、150円やるからその席空けろと}。 
吉野家ってのはな、もっと殺伐としてるべきなんだよ。 
Uの字テーブルの向かいに座った奴といつ喧嘩が始まってもおかしくない、 
刺すか刺されるか、そんな雰囲気がいいんじゃねーか。女子供は、すっこんでろ。 
で、やっと座れたかと思ったら、隣の奴が、大盛つゆだくで、とか言ってるんです。 
そこでまたぶち切れですよ。 
あのな、つゆだくなんてきょうび流行んねーんだよ。ボケが。 
得意げな顔して何が、つゆだくで、だ。 
お前は本当につゆだくを食いたいのかと問いたい。問い詰めたい。小1時間問い詰めたい。 
お前、つゆだくって言いたいだけちゃうんかと。 
吉野家通の俺から言わせてもらえば今、吉野家通の間での最新流行はやっぱり、 
ねぎだく、これだね。 
大盛りねぎだくギョク。これが通の頼み方。 
ねぎだくってのはねぎが多めに入ってる。そん代わり肉が少なめ。これ。 
で、それに大盛りギョク(玉子)。これ最強。 
しかしこれを頼むと次から店員にマークされるという危険も伴う、諸刃の剣。 
素人にはお薦め出来ない。 
まあお前らド素人は、牛鮭定食でも食ってなさいってこった。 


そんな事より、きいてくれる? 
このあいだ、ハローワーク行ったんです。職安。 
そしたらなんか人がめちゃくちゃいっぱいで入れないんです。 
で、よく見たらなんか垂れ幕下がってて、失業率過去最高、とか書いてあるんです。 
もうね、アホかと。馬鹿かと。 
お前らな、失業率5%如きで普段来てない職安に来てんじゃねーよ、ボケが。 
5%だよ、5%。 
なんか女連れとかもいるし。カップル2人で職探しか。おめでてーな。 
よーし俺ココ面接受けちゃうぞー、とか言ってるの。もう見てらんない。 
お前らな、履歴書やるからその列空けろと。 
失業者ってのはな、もっと殺伐としてるべきなんだよ。 
公園に寝転がってるルンペンといつ喧嘩が始まってもおかしくない、 
刺すか刺されるか、そんな雰囲気がいいんじゃねーか。女・ドキュソは、すっこんでろ。 
で、やっと失業給付受け取ったと思ったら、隣の奴が、いや~リストラされちゃって、とか言ってるんです。 
そこでまたぶち切れですよ。 
あのな、リストラなんてきょうび流行んねーんだよ。ボケが。 
得意げな顔して何が、リストラされちゃって、だ。 
お前は本当に働きたいのかと問いたい。問い詰めたい。小1時間問い詰めたい。 
お前、不幸自慢したいだけちゃうんかと。 
失業通の俺から言わせてもらえば今、失業通の間での最新流行はやっぱり、 
ヒッキー、これだね。 
童貞ヒッキーアニヲタ。これが通の生き方。 
ヒッキーってのは時間が有り余ってる。そん代わりやる気が少なめ。これ。 
で、それに童貞アニヲタ(デブ)。これ最強。 
しかしこれにハマルと次から両親にマークされるという危険も伴う、諸刃の剣。 
素人にはお薦め出来ない。 
まあ、日雇い労働者でもやってなさいってこった。 

\section{吉野家コピペ・英語}

By the way, Please listen to me. 
I went to YOSHINOYA yesterday. 
\textbf{But YOSHINOYA was so crowded}
that I couldn't find any seat to sit.
\emph{I looked over YOSHINOYA and found "150off" ad}. 
I was disgusted! "U, MOTHERFUCKER?! BASTARD?!" 
"It's fuckin' crazy of them who ain't regular customers 
to come to YOSHINOYA for 150 円." 
"150yen! only 150YEN!" 
Even a family came to YOSHINOYA. "Did all 4 of your family come here ? 
U, complete Idiot!" 
I saw their dad tell his children,"Yes! I'll order a TOKU-MORI!" 
I couldn't stand anymore. 
I wanted to say to them,"Get out of my sight and I'll give you 円 150." 
In my opinion, YOSHINOYA should have a brural atmosphere. 
In YOSHINOYA, it should be natural whenever customers 
sitting down across the U-counter start the fight. 
The tenson in which I might be stabbed or stab must be pleasant. 
"Women and Children, get lost!" 
Finally, I could get my seat. But at that moment, 
the fool sittin' next to me ordered "O-MORI with TUYUDAKU". 
I got furious again. 
"Hey, TUYUDAKU will never be in fashion nowadays. 
Kiss my ass! 
What on hell makes you order a TUYUDAKU with such a proud face? 
I wanna ask U if U really wanna eat a TUYUDAKU. 
I wanna ask severely. 
I wanna ask thoroughly for one hour. 
U just wanna say TUYUDAKU, don't U? 
In my view as an authority on YOSHINOYA, 
the latest trend among the authorities is exactly "NEGIDAKU". 
We can think nothing but this. 
"O-MORI,NEGIDAKU,GYOKU!" This is the way we authorities order. 
NEGIDAKU is the GYU-DON that has more onion but less beef instead. 
What's more, "O-MORI and GYOKU". These are strongest. 
But if you order this combination, you'll be exposed to the danger of 
staff's caution. This cuts both ways. 
I never recommend beginner this. 
Anyway, why don't U beginner have a GYU-JYAKE TEISYOKU? 




\end{document}